\documentclass[12pt]{article}

\usepackage[utf8]{inputenc}
\usepackage[brazil]{babel}
\usepackage[a4paper,left=3cm, right=2cm,top=2.5cm, bottom=2.5cm]{geometry}
\usepackage{amsmath}
\usepackage{graphicx}
\usepackage{float}
\usepackage{multirow}
\usepackage{authblk}
\usepackage{fancyhdr}
\usepackage{xcolor}
\usepackage{cite}


\title{\textbf{ENG1456 - Algoritmos Genéticos - Trabalho 1}}
\author{\textbf{Aluno: Matheus Carneiro Nogueira - 1810764}}
\affil{}
\author{\textbf{Professora: Karla Figueiredo}}
\affil{}
\pagestyle{fancy}
\fancyhf{}
\lhead{{\small \textcolor{gray}{PUC-Rio ENG1456}}}
\renewcommand{\headrulewidth}{0pt}
\date{}
\renewcommand{\footrulewidth}{0pt}
\fancyfoot[C]{\thepage}

\begin{document}
	\maketitle
	\tableofcontents
	
	
	\begin{abstract}
		Este documento consiste no relatório do trabalho 1 do módulo de Algoritmos Genéticos da disciplina ENG1456 da PUC-Rio. O objetivo deste trabalho é estudar diferentes modelos de Algoritmos Genéticos para a tentativa de otimização da função $F6$ apresentada em sala. Foi utilizado o programa \textit{ICADEMO} para gerar os modelos pedidos nos enunciados. Em todas as figuras do ICADEMO em que há mais de uma curva plotada, as configurações são referentes apenas à última curva. Foram consultados os materiais de aula, o livro \cite{davis1991handbook} e outros materiais devidamente referenciados.
	\end{abstract}
	
\section{Reproduzindo Resultados}

\textbf{Enunciado:}
Variando os parâmetros, execute Algoritmos Genéticos de modo a obter resultados semelhantes aos apresentados no livro texto. Os parâmetros usados no livro se encontram na tabela abaixo. Compare as curvas referentes à média de 20 rodadas de cada GA. Incluir dois gráficos: um com GA1-1, GA2-1 e GA2-2 e outro com GA 2 -3 e GA2-4. Utilizar somente one-point-crossover.

\begin{figure}[H]
	\centering
	\includegraphics[width=0.9\linewidth]{Imagens/tabela_especificacao_modelos}
	\caption{Tabela com as especificações dos modelos}
	\label{fig:tabelaespecificacaomodelos}
\end{figure}


As duas imagens abaixo exibem os 5 GA's solicitados no enunciado. Em ambas as figuras também foi plotada a curva da busca aleatória para fins de comparação.

\begin{figure}[H]
	\centering
	\includegraphics[width=0.7\linewidth]{Imagens/questao1_1}
	\caption{Experimento 1: GA=1-1, Norm. Linear: GA=2-1, Elitismo: GA=2-2}
	\label{fig:questao11}
\end{figure}

\begin{figure}[H]
	\centering
	\includegraphics[width=0.7\linewidth]{Imagens/questao1_2}
	\caption{Steady State 1: GA=2-3, SS s/ duplicados: GA=2-4}
	\label{fig:questao12}
\end{figure}

Podemos, facilmente, perceber alguns detalhes imporantes. Primeiramente, o GA 1-1 da figura\ref{fig:questao11} se mostrou pior que a busca aleatória. Isso não é surpresa,  uma vez que esse modelo não usufrui de nenhum operador além do crossover e mutação. AO analisar as demais curvas que utilizam esses recursos, é de se esperar que o resultado seja melhor que uma busca aleatória, senão não haveria motivo para utilizar um algoritmo genético.

Uma simples normalização linear já foi suficiente para que o GA 2-1 apresentasse melhor resultado que a busca aleatória.

A utilização do elitismo é importante por garantir que, durante a evolução, o melhor indivíduo de $t+1$ seja sempre melhor ou igual ao melhor de $t$. Isso se percebe pelo fato da curva cinza da figura \ref{fig:questao11} ser monotônica, isto é, sempre crescente. Esse fato é suficiente para explicar o melhor desempenho desse modelo em relação aos anteriores.

Ambos os Steady States foram utilizados com GAP=80\%, o que quer dizer que \textbf{COMPLETAR COM INTERPRETAÇÃO DO GAP E COM DIFERENÇA DE COM E SEM DUPLICADOS}

Comparando as duas figuras, podemos perceber que, dentre os 5 GA's apresentados e a busca aleatória, aquele que utiliza elitismo junto com normalização linear (GA2-2) apresenta o melhor resultado.

\section{GAP ideal}
\textbf{Enunciado:} Para os GAs que utilizam steady-state, determine o GAP (número de indivíduos substituídos a cada ciclo) ideal. Para isso, use um incremento de 5 indivíduos a cada tentativa, começando com um GAP=5. Não entregue os gráficos referentes aos testes de GAP.\\

O melhor resultado para cada GAP é exibido nas tabelas as seguir.

\begin{table}[H]
	\centering
	\begin{tabular}{|c|c|c|c|c|c|c|c|c|}
		\hline
		GAP & 5 & 10 & 15 & 20 & 25 & 30 & 35 & 40 \\
		\hline
		Num de 9's &  &  &  &  &  &  &  &  \\
		\hline
		GAP & 45 & 50 & 55 & 60 & 65 & 70 & 75 & 80 \\
		\hline
		Num de 9's &  &  &  &  &  &  &  &  \\
		\hline
		GAP & 85 & 90 & 95 & 100 &  &  &  &  \\
		\hline
		Num de 9's &  &  &  &  &  &  &  &  \\
		\hline
	\end{tabular}
	\caption{Tabela com melhor resultado para cada GAP Steady State com duplicados}
\end{table}

\begin{table}[H]
	\centering
	\begin{tabular}{|c|c|c|c|c|c|c|c|c|}
		\hline
		GAP & 5 & 10 & 15 & 20 & 25 & 30 & 35 & 40 \\
		\hline
		Num de 9's &  &  &  &  &  &  &  &  \\
		\hline
		GAP & 45 & 50 & 55 & 60 & 65 & 70 & 75 & 80 \\
		\hline
		Num de 9's &  &  &  &  &  &  &  &  \\
		\hline
		GAP & 85 & 90 & 95 & 100 &  &  &  &  \\
		\hline
		Num de 9's &  &  &  &  &  &  &  &  \\
		\hline
	\end{tabular}
	\caption{Tabela com melhor resultado para cada GAP Steady State sem duplicados}
\end{table}

Como explicado na seção anterior, GAP significa a porcentagem da população que será trocada. Logo, um GAP de 0\% significaria que nenhuma parcela da população seria trocada, ou seja, não haveria Steady State. 

\textbf{COMPLETAR INTERPRETAÇÃO}
	
\section{Taxas Crossover e Mutação}
\textbf{Enunciado:}
Verifique o que acontece quando se roda o GA2-1 20 vezes com taxa de crossover muito baixa (pouca recombinação em torno de 10\%) e alta taxa de mutação (muitas mudanças aleatórias em torno de 80\%). Imprima o resultado (um gráfico), compare com o resultado do GA2-1 obtido no item 1 e explique
brevemente o que acontece.\\

A figura \ref{fig:questao3} abaixo exibe o GA com as taxas pedidas.

Sabemos que a altas taxas de mutação aumentam a aleatoriedade do processo de evolução e que baixas taxas de crossover tornam a busca pela seleção ótima mais lenta, uma vez que há menor combinação dos progenitores.

Ao analisar a figura \ref{fig:questao3} podemos perceber uma alta volatilidade na qualidade da solução, isto é, grnade variação entre picos e vales. Esse fato é resultado da alta taxa de mutação, uma vez que ela atrapalha a evolução normal do algoritmo ao adicionar aleatoriedade. A baixa taxa de mutação, por sua vez, é percebida de forma mais sutil pela falta de caráter crescente da curva. Como são feitos poucos cruzamentos, utiliza-se pouco os bons progenitores para gerar bons filhos, o que diminui a qulidade da evolução da solução. 

\begin{figure}[H]
	\centering
	\includegraphics[width=0.7\linewidth]{Imagens/questao3}
	\caption{GA2-1 com taxas de crossover 10\% e mutação 80\%}
	\label{fig:questao3}
\end{figure}


\section{Tamanho da População}
\textbf{Enunciado:}
Analise o efeito do tamanho da população, obtendo as curvas de desempenho do GA2-2 (20 rodadas) para vários tamanhos de população (ex: 20, 50, 100, 150) e sempre com o mesmo número de gerações (total de indivíduos variável). Imprima as curvas para e tire conclusões sobre o efeito do tamanho da população no desempenho do
algoritmo genético.\\

Antes de partirmos para a avaliação dos diferentes GA's, devemos lembrar como definir o tamanho da população em relação do número de gerações e do número total de indivíduos(avaliações).

Esses parâmetros relacionam-se de acordo com a seguinte expressão:

\begin{equation*}
	avaliacoes=num\_geracoes\times tamanho\_populacao
\end{equation*}

A partir dessa relação e dos valores solicitados pelo enuncido para os tamanhos de população, mantendo o número de gerações igual a 40, chegamos aos seguintes valores:

\begin{table}[H]
	\centering
	\begin{tabular}{|c|c|c|c|c|}
		\hline
		tam\_pop & 20 & 50 & 100 & 150 \\
		\hline
		avaliações & 800 & 2000 & 4000 & 6000 \\
		\hline
	\end{tabular}
\end{table}

A figura abaixo exibe os resultados para esses valores de população.

\begin{figure}[H]
	\centering
	\includegraphics[width=0.7\linewidth]{Imagens/questao4}
	\caption{Elitismo (0) = tam\_pop=20; Elitismo (1) = tam\_pop=50; Elitismo (2) = tam\_pop=100; Elitismo (3) = tam\_pop=150;}
	\label{fig:questao4}
\end{figure}

Podemos, a partir da análise da imagem, concluir que uma população de 20 indivíduos com 40 gerações é pouco para a evolução do algoritmo. O que nos indicar esse fato é a curva amarela da figura \ref{fig:questao4}, que representa a população de 50, ser praticamente a mesma da curva azul, população de 20, até o número de avaliações existentes na curva azul. Isso nos indica que, se houver mais do que 20 indivíduos por população, o algoritmo seria capaz de evoluir mais em direção à solução ótima. 

Por outro lado, para as populações de 50, 100 e 150, o resultado final da soluçao é muito similar, ficando em torno de 2.1 noves. Isso nos revela que aumentar o número de indivíduos da população mantendo a mesma quantidade de gerações não infere, necessariamente, na melhora da solução. Podemos justificar isso da seguinte maneira: mais indivíduos por população aumenta o paralelismo da busca, mas, em dado momento, o algoritmo já possui indivíduos suficientes buscando a solução ótima e precisaria, apenas, de mais tempo (gerações) para encontrá-la. 

\section{Convergência}
\textbf{Enunciado:}
Repita o GA2-1 e o GA2-2 (20 rodadas cada) modificando apenas o total de indivíduos criados para o 10000. Imprima as curvas em dois um gráficos separados, um para o GA2-1 e outro para o GA2-2, e verifique se é vantajoso todo esse esforço computacional, em outras palavras, determine o número de
indivíduos para o qual cada algoritmo converge.

A figura abaixo exibe os resultados dos GA's solicitados no enunciado.

\begin{figure}[H]
	\centering
	\includegraphics[width=0.7\linewidth]{Imagens/questao5_1}
	\caption{GA 2-1 com avaliações de 4000 (azul) e 10000 (amarelo)}
	\label{fig:questao51}
\end{figure}

\begin{figure}[H]
	\centering
	\includegraphics[width=0.7\linewidth]{Imagens/questao5_2}
	\caption{GA 2-2 com avaliações de 4000 (azul) e 10000 (amarelo)}
	\label{fig:questao52}
\end{figure}

A análise de convergência dos GA's fornece conclusões distintas para o caso com e sem elitismo. Comecemos pelo GA 2-1, com normalização linear e sem elitismo.

Nesse caso, percebe-se pela figura \ref{fig:questao51} que não há uma convergência para o mesmo valor de avaliação, isto é, o algoritmo com mais avaliações (curva amarela) apresentou um desempenho melhor que o com menos (curva azul). Isso pode ser explicado pelo fato de, em elitismo, não garantirmos que a curva de avaliação será sempre crescente. Sendo assim, precisamos de mais tempo, ou seja, mais avaliações e gerações, para chegar a um resultado mais próximo do ótimo, o que é evidenciado pela diferença entre as curvas.

Para o GA 2-2, com elitismo e normalização linear, a conclusão é outra. Com a utilização do elitismo, percebemos convergência para o mesmo valor de número de noves. Note que a curva amarela, após alcançar o valor mais alto da curva azul (2.6), permanece praticamente constante. Com esse operador, geramos uma curva monotônica e convergimos para um valor bom mais rapidamente. Isso nos mostra que aumentar a população de um GA com elitsmo não gera melhores resultados, uma vez que esse operador já é suficiente para produzir uma solução boa. Aumentar o esforço computacional, nesse caso, é desncessário e, dito isso, não recomendável.

\section{Crossover}
\textbf{Enunciado:}
Compare o efeito dos 3 tipos de crossover disponíveis na ferramenta, executando o GA2-1 (s/ elitismo) e o GA2-2 (c/elitismo) com apenas 2500 indivíduos (20 rodadas) para cada tipo de crossover, usando taxa de crossover 80\%. Imprima as curvas em dois um gráficos separados , um para o GA2-1 e outro para o GA2-2, e tire conclusões a respeito da característica conservadora/destrutiva de cada crossover.

As figuras abaixo exibem os gráficos para os 3 tipos de crossover para os GA's 2-1 e 2-2.

\begin{figure}[H]
	\centering
	\includegraphics[width=0.7\linewidth]{Imagens/questao6_1}
	\caption{GA 2-1; Norm Linear (0) = crossover de um ponto ; Norm Linear (1) = crossover de dois pontos ; Norm Linear (2) = crossover uniforme}
	\label{fig:questao61}
\end{figure}
\begin{figure}[H]
	\centering
	\includegraphics[width=0.7\linewidth]{Imagens/questao6_2}
	\caption{GA 2-2; Elitismo (0) = crossover de um ponto ; Elitismo (1) = crossover de dois pontos ; Elitismo (2) = crossover uniforme}
	\label{fig:questao62}
\end{figure}


\section{Normalização Linear}
\textbf{Enunciado:}
Repita o GA2-3COM gap = 75 para vários valores de máximo. Verifique o que acontece quando o valor de máximo aumenta e diminui (avalie para os valores 10, 50, 100, 200, 300). Imprima as curvas em apenas um gráfico e tire breves conclusões .

\begin{figure}[H]
	\centering
	\includegraphics[width=0.7\linewidth]{Imagens/questao7}
	\caption{Steady State (0) = max 10; Steady State (1) = max 50; Steady State (2) = max 100; Steady State (3) = max 200; Steady State (4) = max 300;}
	\label{fig:questao7}
\end{figure}


\section{Gerais}
\textbf{Enunciado:}
Fazendo variações nos parâmetros e técnicas disponíveis no GADEMO, estude livremente o efeito de cada umdestes no desempenho de algoritmos genéticos. Destaque e explique uma importante constatação.

	\bibliography{bibliografia} 
	\bibliographystyle{plain}
\end{document}