\documentclass[12pt]{article}

\usepackage[utf8]{inputenc}
\usepackage[brazil]{babel}
\usepackage[a4paper,left=3cm, right=2cm,top=2.5cm, bottom=2.5cm]{geometry}
\usepackage{amsmath}
\usepackage{graphicx}
\usepackage{float}
\usepackage{multirow}
\usepackage{authblk}
\usepackage{fancyhdr}
\usepackage{xcolor}

\title{\textbf{ENG1456 - Redes Neurais - Trabalho 1 Classificação}}
\author{\textbf{Aluno: Matheus Carneiro Nogueira}}
\affil{}
\author{\textbf{Professora: Marley}}
\affil{}
\pagestyle{fancy}
\fancyhf{}
\lhead{{\small \textcolor{gray}{PUC-Rio ENG1456}}}
\renewcommand{\headrulewidth}{0pt}
\date{}
\renewcommand{\footrulewidth}{0pt}
\fancyfoot[C]{\thepage}

\begin{document}
	\maketitle
	\tableofcontents
	
\begin{abstract}
	Este documento consiste no relatório do trabalho 1 do módulo de Redes Neurais da disciplina ENG1456 da PUC-Rio. Nele será explicada a implementação de modelos de Redes Neurais MLP para a clafficicação do dataset Ionosphere, disponibilizado pela professora da disciplina. Esta classificação é binária e visa separar as antenas do dataset em "good" ou "bad". A seções do relatório são definidas de acordo com as perguntas principais que constam no arquivo Guia de Atividades.
\end{abstract}

\section{Compreemsão do problema e análise de variáveis}

\subsection{Observe a base de dados do problema. Existem variáveis que podem ser	eliminadas do dataset? Justifique.}

\subsection{Implemente técnicas de visualização de dados e seleção de variáveis para extrair características importantes sobre a base de dados. Explique	a motivação destas técnicas e o que é possível inferir dos resultadosobtidos.}


\section{Treinamento do modelo de Rede Neural}

\subsection{Com as configurações do modelo MLP previamente definidas no script, faça o treinamento da Rede Neural sem normalizar os atributos	numéricos. Comente o resultado obtido, baseado nas métricas de	avaliação disponíveis (acurácia, precision, recall, F1-Score, Matriz de	confusão, etc.)}

\subsection{Agora normalize os dados de entrada e treine novamente o modelo MLP.	Avalie os resultados obtidos e comente o efeito da normalização no	treinamento da Rede Neural.}
	
\section{Mudança de configurações do modelo}

\subsection{Insira o conjunto de validação para o treinamento do modelo. Avalie o resultado obtido.}

\subsection{Modifique o tempo de treinamento (épocas) da Rede Neural. Escolha dois valores distintos (e.g. 1 e 1000 épocas) e avalie os resultados.}

\subsection{Modifique a taxa de aprendizado da Rede Neural. Escolha dois valores distintos (e.g. 0,001 e 0,1) e avalie os resultados.}

\subsection{Modifique a quantidade de neurônios na camada escondida da RedecNeural. Escolha dois valores distintos (e.g. 2 e 70 neurônios) e avalie os	resultados.}


\section{Teste Livre}

\subsection{Faça novos testes para avaliar o desempenho da Rede Neural no	problema designado. Use a técnica K-Fold (com K = 10) para analisar o	resultado obtido.}

\subsection{Faça análises e novas implementações que você julgue importante para o seu trabalho. Não esqueça de explicar a motivação da análise realizada.}

\end{document}